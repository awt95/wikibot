\chapter{Implementation}
\label{ch:implementation}
\section{Introduction}
This chapter describes the planning and implementation phases.

\section{Software Development Methodology}
\subsection{Waterfall Model}
\subsection{}

\section{Planning}
As the timeline of the project is restricted and has a fixed deadline, it is important that the project is planned carefully in order to deliver a complete solution.

Gantt


\section{Version Control}
Git - branches

\section{Dependency Management}

\section{Implementation Process}
Following the Gantt chart, ...

\subsection{Initialisation}
The first step of the implementation was to initialise a Git repository and a Java environment with Maven dependency management. Since the project would depend on many libraries for displaying and processing data, Maven is able to manage versioning and interdependencies of these libraries.

\subsection{Program AB}
As a result of the research in earlier phases, it was decided that this system would use a rule-based chatbot architecture, namely AIML as it is widely used in current chatbot systems. Program AB is a Java implementation of AIML 2.0 provided by ALICE A.I. Foundation \cite{programab_2013}. While this repository is no longer being developed, the choice for this project was to use a fork of this source, which is provided as a Maven dependency \cite{lumenrobot2016}.

Once the Maven dependency had been loaded into the project's {\it{pom.xml}} file, I was able to experiment with the chatbot. The {\it{program-ab}} source contains the original AIML implementation of a bot named S.U.P.E.R. created by \citeauthor{wallace2009anatomy}, which gives examples of the many elements of the AIML style.


\subsection{Web Interface}

\subsection{DBPedia}

\subsection{AIML}

\subsection{Hibernate Database}



\section{Challenges}
\subsection{Spring Framework}

\subsection{Dataset Inconsistencies}
\begin{itemize}
	\item Actors 
	\item Countries - England displayed as a dbo:MusicalArtist
\end{itemize}

\subsection{Limitations of AIML}
Regex
Pattern matching - rule-based - synonyms

\subsection{SPARQL Queries}
\begin{itemize}
	\item Searching
	\item Lists - return order - vrank
\end{itemize}
\section{Conclusion}


