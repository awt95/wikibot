\chapter{Introduction}
\label{ch:intro}
\section{Overview}
This project 

\section{Background}

\section{Motivation}
My personal motivation for this project derives primarily from my placement year during my studies at university.  

\newpage
\section{Aims and Objectives}
\label{sec:aims}
The primary aim of the project is to investigate the increasingly prevalent uses of chatbots in industry and research. Additionally, this project will explore the numerous technologies used for implementing a chatbot which queries a knowledge base and evaluate the feasibility of using these technologies in chatbot development projects. Finally, the project will aim to implement a successful system such that it may be adapted to gather information from additional datasets in the future. The following list outlines the measurable objectives of the project, against which the success of the project will be assessed.

\begin{itemize}
	\itemsep0em 
	\item Discuss and explore current and future uses of chatbots in research and industry environments.
	\item Gather and discuss user requirements for a chatbot knowledge base implementation within the scope of the project.
	\item Implement a novel chatbot application to resolve user queries and provide worthwhile responses automatically.
	\item Test the implementation against the requirements set out by the users.
	\item Evaluate the success of the system and the implementation of the technologies with regards to the scope and requirements of the project, and provide suggestions for future development.

\end{itemize}

\section{Success Criteria}

\newpage
\section{Outline of Chapters}
This section describes the contents and purpose of each chapter of this report.

\subsection*{Introduction}
This chapter outlines an overview of the project, and describes the aims and objectives for this project; these aims and objectives will be used to measure the success of the project in the evaluation.

\subsection*{Literature Review}
This chapter explores the concepts and technologies used in chatbots and machine learning. Various machine learning concepts are explored and compared, as well as data sources and programming languages that can be used to implement a system. Existing solutions and related works are analysed.

\subsection*{Analysis}
In this chapter, a set of requirements for the system is built based on input from a number of users. The requirements are outlined and prioritised. Methodologies for implementing the project are analysed and discussed in preparation for designing and developing a successful solution.

\subsection*{Design}
This chapter details the architecture of the proposed system, using a number of UML diagrams. Design and implementation methodologies are justified for the scope of this project.

\subsection*{Implementation}
This chapter examines some of the difficulties and experiences faced during the implementation of this project. These are then discussed further in Chapter~\ref{ch:conclusion}.

\subsection*{Testing}
This chapter documents the testing phase of the project. A number of testing strategies are discussed and implemented. The results of the testing are analysed in this section.

\subsection*{Conclusion}
This summarises the overall successes of the project, by considering the initial objectives, the requirements of the system, and the result of the testing. The overall strategy of the project is considered in hindsight. Potential future work is discussed, and a final statement for the project is given.

\subsection*{Ethics}
This section explores potential legal, social, ethical and professional (LSEP) implications of the project. The chapter discusses ethical considerations surrounding the project, inline with guidelines provided by the University of Surrey \cite{surreyethics} and the British Computer Society (BCS) \cite{bcs2019conduct}.

\subsection*{Appendix}
Supplementary documents and addenda are appended to the document in this section.
\begin{itemize}
	\item Appendix A - WikiMedia API
	\par This appendix displays the server response from the WikiMedia API. The impracticality of this method is discussed in Appendix~\ref{subsec:candidates}.
	
	\item Appendix B - Chatting with Mitsuku
	\par This item logs a conversation between me and Mitsuku. This took place during my research into existing solutions, as explained and analysed in Section~\ref{subsec:Mitsuku}.
	
	\item Appendix C - SAGE Form
	\par This document is the result of executing the University of Surrey Self-Assessment for Governance and Ethics (SAGE) as a requirement of the project. This self-assessment provides indication of whether further ethical review is requirement for the project; in this case further action is not required. A further discussion of this is given in \ref{ch:ethics}.
\end{itemize}
